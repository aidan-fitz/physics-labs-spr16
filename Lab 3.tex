%%%%%%%%%%Packages%%%%%%%%%

\documentclass[11pt, titlepage, letterpaper, twoside]{article}
\usepackage{amsmath, amsthm, amssymb}
\usepackage{hyperref, pgf, tikz}
\usepackage{fancyhdr}
\usetikzlibrary{arrows}
\usepackage[margin=1.25in]{geometry}

%%%%%%%%%%%%%%%%%%%%%%%%%%%
\usetikzlibrary{circuits,circuits.ee.IEC}
\usepackage{circuitikz}
\tikzset{circuit declare symbol = amm}
\tikzset{set amm graphic ={draw,generic circle IEC, minimum size=7mm,info=center:A}}
\tikzset{circuit declare symbol = voltm}
\tikzset{set voltm graphic ={draw,generic circle IEC, minimum size=7mm,info=center:V}}


%%%%%%%%%%New Commands%%%%%%%%%



%%%%%%%%%%%%%%%%%%%%%%%%%%%
\pagestyle{fancy}
\frenchspacing

\lhead{Lab \#3 }  %insert lab # here
\rhead{\thepage}
\cfoot{}

\title{\textbf{RC Circuit} \\ \ \\ \large Lab \#3 }
\author{Name: Aidan Fitzgerald \\ Partner: Jared Beh}
\date{June 7, 2016}
\begin{document}

\maketitle

\begin{center}
\LARGE RC Circuit
\end{center}

\section*{Objective}
Infer the relationship between the time constant $\tau$, resistance $R$, and capacitance $C$ of an RC circuit.

\section{Introduction}
An RC circuit is a type of circuit made of a resistor and a capacitor connected in series, like so:

\begin{figure}[h!]
  \centering
  \begin{circuitikz}[circuit ee IEC] \draw
    (0,0) -- (0,6) -- (6,6) to[short,-*] (6,3)
    (6,1.5) to[short,*-] (6,0) to[battery,l=$V_0$] (0,0)
    (0,2.25) to[short,*-] (1,2.25) to[C,l=$C$] (3,2.25) to[R,l=$R$] (5.25,2.25) to[short,-*] (6,2.25)
    (1,2.25) to[short,*-] (1,4) to[voltm] (3,4) to[short,-*] (3,2.25)
    ;
    \draw[very thick] (6,2.25) -- +(30:0.75);
  \end{circuitikz}
  \caption{RC circuit}
\end{figure}


When a constant DC voltage $V_0$ is applied to the circuit, an electric field builds up inside the capacitor
as it gradually charges. By Kirchhoff's voltage law, the circuit's behavior as a function of time is given
by the first-order differential equation

\begin{equation}
  V_0 - \frac{Q}{C} - R\dot{Q} = 0
\end{equation}

The solution to this equation is

\begin{equation}
  V(t) = V_0\,(1 - e^{-t/RC})
\end{equation}

As $t$ approaches infinity, $V(t)$ approaches $V_0$.

The time constant $\tau$ of an RC circuit is defined such that

\begin{equation}
V(\tau) = V_0\,(1 - e^{-1}) \approx 0.63\,V_0.
\end{equation}

Therefore,

\begin{equation}
  \tau = RC.
\end{equation}

Note that $\tau$ does not depend on $V_0$: the greater the applied voltage, the faster the capacitor charges.

When a capacitor is discharging into an RC circuit, it produces an exponentially decaying direct current. As a function of time,
this is

\begin{equation}
  V(t) = V_0\, e^{-t/\tau}
\end{equation}

We can graph the logarithm of voltage as a linear function of time. From Eq. 2, we obtain

\begin{equation}
  \ln(V_0 - V) = -\frac{t}{RC} + \ln V_0
\end{equation}

and from Eq. 5, we get

\begin{equation}
  \ln V = -\frac{t}{RC} + \ln V_0
\end{equation}

\section{Procedures and Results}
We set up the circuit shown in Figure 1.

\section{Discussion}
You

\section{Conclusion}
Today

\end{document}
