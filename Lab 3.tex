%%%%%%%%%%Packages%%%%%%%%%

\documentclass[11pt, titlepage, letterpaper, twoside]{article}
\usepackage{amsmath, amsthm, amssymb}
\usepackage{hyperref, pgf, tikz}
\usepackage{fancyhdr}
\usetikzlibrary{arrows}
\usepackage[margin=1.25in]{geometry}

%%%%%%%%%%%%%%%%%%%%%%%%%%%
\usetikzlibrary{circuits,circuits.ee.IEC}
\usepackage{circuitikz}
\tikzset{circuit declare symbol = amm}
\tikzset{set amm graphic ={draw,generic circle IEC, minimum size=7mm,info=center:A}}
\tikzset{circuit declare symbol = voltm}
\tikzset{set voltm graphic ={draw,generic circle IEC, minimum size=7mm,info=center:V}}


%%%%%%%%%%New Commands%%%%%%%%%



%%%%%%%%%%%%%%%%%%%%%%%%%%%
\pagestyle{fancy}
\frenchspacing

\lhead{Lab \#3 }  %insert lab # here
\rhead{\thepage}
\cfoot{}

\title{\textbf{RC Circuit} \\ \ \\ \large Lab \#3 }
\author{Name: Aidan Fitzgerald \\ Partner: Jared Beh}
\date{June 7, 2016}
\begin{document}

\maketitle

\begin{center}
\LARGE RC Circuit
\end{center}

\section*{Objective}
Infer the relationship between the time constant $\tau$, resistance $R$, and capacitance $C$ of an RC circuit.

\section{Introduction}
An RC circuit is a type of circuit made of a resistor and a capacitor connected in series, like so:

\begin{figure}[h!]
  \centering
  \begin{circuitikz}[circuit ee IEC] \draw
    (0,0) -- (0,6) -- (6,6) to[short,-*] (6,3)
    (6,1.5) to[short,*-] (6,0) to[battery,l=$V_0$] (0,0)
    (0,2.25) to[short,*-] (1,2.25) to[C,l=$C$] (3,2.25) to[R,l=$R$] (5.25,2.25) to[short,-*] (6,2.25)
    (1,2.25) to[short,*-] (1,4) to[voltm] (3,4) to[short,-*] (3,2.25)
    ;
    \draw[very thick] (6,2.25) -- +(30:0.75);
  \end{circuitikz}
  \caption{RC circuit}
\end{figure}


When a constant DC voltage $V_0$ is applied to the circuit, an electric field builds up inside the capacitor
as it gradually charges. By Kirchhoff's voltage law, the circuit's behavior as a function of time is given
by the first-order differential equation

\begin{equation}
  V_0 - \frac{Q}{C} - R\dot{Q} = 0
\end{equation}

The solution to this equation is

\begin{equation}
  V(t) = V_0\,(1 - e^{-t/RC})
\end{equation}

As $t$ approaches infinity, $V(t)$ approaches $V_0$.

The time constant $\tau$ of an RC circuit is defined such that

\begin{equation}
  V(\tau) = V_0\,(1 - e^{-1}) \approx 0.63\,V_0.
\end{equation}

\pagebreak

Therefore,

\begin{equation}
  \tau = RC.
\end{equation}

Note that $\tau$ does not depend on $V_0$: the greater the applied voltage, the faster the capacitor charges.

When a capacitor is discharging into an RC circuit, it produces an exponentially decaying direct current. As a function of time,
this is

\begin{equation}
  V(t) = V_0\, e^{-t/\tau}
\end{equation}

Substituting $t = \tau$,

\begin{equation}
  V(\tau) = V_0\, e^{-1} \approx 0.37\,V_0.
\end{equation}


\section{Procedures and Results}
We set up the circuit shown in Figure 1. We set $V_0$ to 4.5 V, $R$ to $1.6\,\mathrm{k\Omega}$, and $C$ to $27\,\mathrm{\mu F}$. We turned on
the power and took voltage measurements every 5 seconds for 45 seconds.

\begin{table}[h!]
\centering
\caption{Charging from a 4.5-V DC power source.}
\label{charging-1}
\begin{tabular}{|l|l|l|}
\hline
Time (s) & Voltage (V) & \% of $V_0$ \\ \hline
0        & 0           & 0.00\%     \\ \hline
5        & 1.01        & 22.44\%    \\ \hline
10       & 2.04        & 45.33\%    \\ \hline
15       & 2.93        & 65.11\%    \\ \hline
20       & 3.27        & 72.67\%    \\ \hline
25       & 3.31        & 73.56\%    \\ \hline
30       & 3.26        & 72.44\%    \\ \hline
35       & 3.22        & 71.56\%    \\ \hline
40       & 3.2         & 71.11\%    \\ \hline
45       & 3.18        & 70.67\%    \\ \hline
\end{tabular}
\end{table}

\pagebreak



Next, we turned off the power and quickly disconnected the DC power source, so that the
capacitor began to discharge. We took voltage measurements every 5 seconds for 60 seconds.

\begin{table}[h!]
\centering
\caption{Discharging after a 45-s charge.}
\label{discharging-1}
\begin{tabular}{|l|l|l|}
\hline
Time (s) & Voltage (V) & \% of $V_0$ \\ \hline
0        & 3.18        & 100.00\%   \\ \hline
5        & 2.39        & 75.16\%    \\ \hline
10       & 2.11        & 66.35\%    \\ \hline
15       & 1.88        & 59.12\%    \\ \hline
20       & 1.69        & 53.14\%    \\ \hline
25       & 1.53        & 48.11\%    \\ \hline
30       & 1.38        & 43.40\%    \\ \hline
35       & 1.24        & 38.99\%    \\ \hline
40       & 1.12        & 35.22\%    \\ \hline
45       & 1.02        & 32.08\%    \\ \hline
50       & 0.92        & 28.93\%    \\ \hline
55       & 0.84        & 26.42\%    \\ \hline
60       & 0.72        & 22.64\%    \\ \hline
\end{tabular}
\end{table}



Then, we repeated these steps with a $200-\Omega$ resistor and $V_0 = 12 \mathrm{V}$.

\begin{table}[h!]
\centering
\caption{Charging with a $200-\Omega$ resistor for 60 s.}
\label{charging-2}
\begin{tabular}{|l|l|l|}
\hline
Time (s) & Voltage (V) & \% of $V_0$ \\ \hline
0        & 0           & 0.00\%     \\ \hline
5        & 6.14        & 51.17\%    \\ \hline
10       & 8.92        & 74.33\%    \\ \hline
15       & 10.33       & 86.08\%    \\ \hline
20       & 10.98       & 91.50\%    \\ \hline
25       & 11.34       & 94.50\%    \\ \hline
30       & 11.53       & 96.08\%    \\ \hline
35       & 11.62       & 96.83\%    \\ \hline
40       & 11.67       & 97.25\%    \\ \hline
45       & 11.71       & 97.58\%    \\ \hline
50       & 11.73       & 97.75\%    \\ \hline
55       & 11.74       & 97.83\%    \\ \hline
60       & 11.75       & 97.92\%    \\ \hline
\end{tabular}
\end{table}

\begin{table}[h!]
\centering
\caption{Discharging for 70 s.}
\label{discharging-2}
\begin{tabular}{|l|l|l|}
\hline
Time (s) & Voltage (V) & \% of $V_0$ \\ \hline
0        & 11.76       & 100.00\%    \\ \hline
5        & 5.05        & 42.94\%     \\ \hline
10       & 2.36        & 20.07\%     \\ \hline
15       & 1.24        & 10.54\%     \\ \hline
20       & 0.67        & 5.70\%      \\ \hline
25       & 0.35        & 2.98\%      \\ \hline
30       & 0.2         & 1.70\%      \\ \hline
35       & 0.13        & 1.11\%      \\ \hline
40       & 0.08        & 0.68\%      \\ \hline
45       & 0.06        & 0.51\%      \\ \hline
50       & 0.04        & 0.34\%      \\ \hline
55       & 0.03        & 0.26\%      \\ \hline
60       & 0.02        & 0.17\%      \\ \hline
65       & 0.02        & 0.17\%      \\ \hline
70       & 0.02        & 0.17\%      \\ \hline
\end{tabular}
\end{table}

\pagebreak

\section{Discussion}

We know from Eq. 3 that during charging, the voltage rises to about 63\% of its maximum level after one time constant.


Similarly, by Eq. 6, during discharging, the voltage drops to about 37\% of its original level after one time constant.



\section{Conclusion}
Today

\end{document}
