%%%%%%%%%%Packages%%%%%%%%%

\documentclass[11pt, titlepage, letterpaper, twoside]{article}
\usepackage{amsmath, amsthm, amssymb}
\usepackage{hyperref, pgf, tikz}

\usepackage{subcaption}

\usepackage{fancyhdr}
\usetikzlibrary{arrows}
\usepackage[margin=1.25in]{geometry}

\usetikzlibrary{circuits,circuits.ee.IEC}
\usepackage{circuitikz}
\tikzset{circuit declare symbol = amm}
\tikzset{set amm graphic ={draw,generic circle IEC, minimum size=7mm,info=center:A}}
\tikzset{circuit declare symbol = voltm}
\tikzset{set voltm graphic ={draw,generic circle IEC, minimum size=7mm,info=center:V}}
%\tikzset{circuit declare symbol = galvm}
%\tikzset{set galvm graphic ={draw,generic circle IEC, minimum size=7mm,info=center:G}}

%%%%%%%%%%New Commands%%%%%%%%%
\pagestyle{fancy}

\frenchspacing



\lhead{Lab \#1 }  %insert lab # here
\rhead{\thepage}
\cfoot{}

\title{\textbf{Measuring Resistance} \\ \ \\ \large Lab \#1 }
\author{Name: Aidan Fitzgerald \\ Partner: Jared Beh}
\date{May 31, 2016}

%%%%%%%%%%%%%%%%%%%%%%%%%%%%%%
\begin{document}

\maketitle

\begin{center}
\LARGE Measuring Resistance
\end{center}

\section*{Objective}
Demonstrate the relative accuracy of the ammeter-voltmeter methods and the Wheatstone bridge method of measuring electrical resistance.

\section{Introduction}

\subsection{Ammeter-Voltmeter Methods}

The resistance of an electrical load can be measured by placing an ammeter in series and a voltmeter in parallel with the load, and dividing the voltage
reading by the current reading, per Ohm's law:

\begin{equation}
R = \frac{V}{I\/}.
\end{equation}

The resistance of an ideal voltmeter $R_v$ is infinite, so that no current passes through it; the resistance of an ideal ammeter $R_a$ is zero. In reality,
however, voltmeters allow a little current to trickle through, and real-world ammeters produce a small voltage drop; these result in discrepancies between
the measured resistance and the true resistance.

The ammeter and voltmeter can be arranged in two different ways:

\begin{figure}[h!]
\centering

\begin{subfigure}[h!]{0.4\textwidth}
\begin{circuitikz}[circuit ee IEC] \draw
  (0,0) to[vR,l=$R_h$] (0,2)
  to[battery,l=$V$] (0,5) -- (2,5)
  to[amm] (2,3.5) to[short,i=$I$] (2,2.4)
  to[R,l=$R$] (2,0) -- (0,0)
  (2,2.5) -- (4,2.5) to[voltm] (4,0) -- (2,0)
;
\end{circuitikz}
\subcaption{Voltmeter parallel to load}
\end{subfigure}
\begin{subfigure}[h!]{0.5\textwidth}
\begin{circuitikz}[circuit ee IEC] \draw
  (0,0) to[vR,l=$R_h$] (0,2)
  to[battery,l=$V$] (0,5) -- (2,5)
  to[amm] (2,3.5) to[short,i=$I$] (2,2.4)
  to[R,l=$R$] (2,0) -- (0,0)
  (2,4.8) -- (4,4.8) to[voltm] (4,0) -- (2,0)
;
\end{circuitikz}
\subcaption{Voltmeter parallel to ammeter and load}
\end{subfigure}

\caption{Two ammeter-voltmeter circuit setups}

\end{figure}

In Figure 1(a), the voltmeter is parallel only to the load, so the voltages across the voltmeter and load are equal and the current gets split between them:
$I_{obs} = I_{true} + I_v$. Rearranging and substituting into Ohm's law, we obtain the following relationship between the true resistances of the load and the
voltmeter:

\begin{equation}
  R = \frac{V}{I_{true}} = \frac{V}{I_{obs} - I_v} = \frac{V}{I_{obs} - V/R_v}
\end{equation}

Because $I_v$ is subtracted from the denominator, the measured value of $R$ is less than the true value.

In Figure 1(b), the voltmeter is parallel to the series combination of the load and the ammeter. The voltages across the voltmeter, load, and ammeter are
related by the equation $V = V_{load} + V_a = IR + IR_a$, so the true resistance is given by subtracting $R_a$ from the measured resistance.

\begin{equation}
  R = \frac{V}{I} - R_a
\end{equation}


\subsection{Wheatstone Bridge Method}

A Wheatstone bridge consists of three resistors of known resistance and a resistor of unknown resistance arranged in a quadrilateral, as shown in Figure 2. Two
opposite endpoints of the quadrilateral are joined by an ammeter (traditionally a galvanometer), and the other two endpoints are connected to the terminals of
a battery.

\begin{figure}[h!]
\centering
\begin{circuitikz}[circuit ee IEC] \draw
  (0,0) to[battery,l=$V$] (0,8)
  to[short,i=$I$] (4,8)
  to[R,l=$R_1$,i=$I_1$] (2,4)
  to[R,l=$R_2$,i=$I_1$] (4,0) -- (0,0)
  (4,8) to[R,l=$R_s$,i=$I_2$] (6,4)
  to[R,l=$R_x$,i=$I_2$] (4,0)
  (2,4) to[amm] (6,4)
  ;
\end{circuitikz}
\caption{Wheatstone bridge}
\end{figure}

When the known resistances are adjusted such that the current through the ammeter is zero, the known and unknown resistances are related by the equation

\begin{equation}
  R_x = \frac{R_2}{R_1} R_s
\end{equation}

\section{Procedures and Results}

\subsection{Ammeter-Voltmeter Methods}

We set up the circuit shown in Figure 1(a), using a rheostat as $R_h$. We adjusted the rheostat to three different settings and took the ammeter and
voltmeter readings.

\begin{table}[h!]
\centering
\caption{Measurements for setup in Figure 1(a)}
\begin{tabular}{|l|l|l|}
\hline
$I$     & $V$       & $R$             \\ \hline
0.084 A & 4.29 V    & $51.07\,\Omega$ \\ \hline
0.068 A & 3.52 V    & $51.76\,\Omega$ \\ \hline
0.055 A & 2.82 V    & $51.27\,\Omega$ \\ \hline
      & Average $R$ & $51.37\,\Omega$ \\ \hline
\end{tabular}
\end{table}

Then we changed the setup to the one shown in Figure 1(b) and took current and voltage measurements again.

\begin{table}[h!]
\centering
\caption{Measurements for setup in Figure 1(b)}
\begin{tabular}{|l|l|l|}
\hline
$I$     & $V$         & $R$             \\ \hline
0.082 A & 4.21 V      & $51.34\,\Omega$ \\ \hline
0.066 A & 3.43 V      & $51.97\,\Omega$ \\ \hline
0.049 A & 2.70 V      & $55.10\,\Omega$ \\ \hline
        & Average $R$ & $52.80\,\Omega$ \\ \hline
\end{tabular}
\end{table}

Finally, we measured the true resistance of the load and the voltmeter by connecting both ends to an ohmmeter.

\begin{table}[h!]
\centering
\caption{True values of $R$ and $R_v$}
\begin{tabular}{|l|l|}
\hline
$R$   & $996\,\mathrm{k\Omega}$ \\ \hline
$R_v$ & $50.7\,\Omega$          \\ \hline
\end{tabular}
\end{table}


\subsection{Wheatstone Bridge Method}

We set up the circuit shown in Figure 2, using a rheostat as $R_s$ and a digital multimeter as the ammeter. We adjusted the rheostat until the current reading
was zero. Then, we measured the values of $R_s$, $R_1$, $R_2$, and $R_x$ by connecting both ends to an ohmmeter.

\begin{table}[h!]
\centering
\caption{True resistances of Wheatstone bridge resistors}
\begin{tabular}{| r | r | r | r |}
  \hline
  $R_s$          & $R_1$          & $R_2$          & $R_x$          \\ \hline
  $10.7\,\Omega$ & $12.5\,\Omega$ & $28.7\,\Omega$ & $27.2\,\Omega$ \\ \hline
\end{tabular}
\end{table}

\section{Discussion}

Taking the true value of $R$ given in Table 3 and the averages of the measurements in Tables 1 and 2, the percent error using the first ammeter-voltmeter
method was 1.32\%, while the percent error for the second was 4.15\%.

To obtain a value of $R_x$ through the Wheatstone bridge method, we substitute the values in Table 4 into Equation 4:

\begin{align*}
  R_x &= \frac{R_2}{R_1} R_s \\
      &= \frac{28.7\,\Omega}{12.5\,\Omega} 10.7\,\Omega \\
      &= 24.6\,\Omega
\end{align*}

The percent error for this method is 9.68\%. This is greater than the percent errors for the two ammeter-voltmeter methods.

\section{Conclusion}
The two ammeter-voltmeter methods differ in that one always yields a value greater than the true resistance while the other always yields a value less than
the true resistance. However, both methods are more precise than the Wheatstone bridge method.

\end{document}
