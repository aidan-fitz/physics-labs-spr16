%%%%%%%%%%Packages%%%%%%%%%

\documentclass[11pt, titlepage, letterpaper, twoside]{article}
\usepackage{amsmath, amsthm, amssymb}
\usepackage{hyperref, pgf, tikz}
\usepackage{fancyhdr}
\usetikzlibrary{arrows}
\usepackage[margin=1.25in]{geometry}

%%%%%%%%%%%%%%%%%%%%%%%%%%%

%%%%%%%%%%New Commands%%%%%%%%%



%%%%%%%%%%%%%%%%%%%%%%%%%%%
\pagestyle{fancy}

\frenchspacing

\lhead{Lab \#6 }  %insert lab # here
\rhead{\thepage}
\cfoot{}

\title{\textbf{Intro to Microwave Optics} \\ \ \\ \large Lab \#6 }
\author{Name: Aidan Fitzgerald \\ Partners: Julia Hou, Ariella Kahan, Jennifer Lee, Jongyoul Lee, Ariel Levy, and Brandon Lin}
\date{}
\begin{document}

\maketitle

\begin{center}
\LARGE Intro to Microwave Optics
\end{center}

\section*{Objective}

Understand how the microwave transmitter and receiver work and use it to demonstrate the properties of electromagnetic radiation.

\section{Introduction}

The microwave optical system uses microwaves to demonstrate the properties of electromagnetic radiation.
This system consists of a transmitter, which emits microwaves in a certain direction, and a receiver,
which measures the intensity of incoming microwaves and displays it on a galvanometer.

We used the microwave optics system to demonstrate the following physical phenomena:

\begin{description}
  \setlength\itemsep{1em}

  \item [Inverse square law] The intensity of an electromagnetic wave at a distance $d$ is proportional to $1/d^2$.

  \item [Reflection] Occurs when a wave bounces off an obstacle. The law of reflection states that the angle of
  incidence is equal to the angle of reflection at every point where the wave touches the obstacle.

  \item [Polarization] Occurs when light waves are ``flattened'' to oscillate in a particular direction perpendicular
  to its direction of propagation.

  \item [Signal distribution] How is the energy of the wave distributed around the transmitter?

\end{description}


\section{Procedures and Results}

\subsection{Inverse square law}

We set up the transmitter and receiver on the arms of a straightened goniometer facing each other. Next, we turned
on the transmitter, dialed up the intensity to 30$\times$, and adjusted both units so that they were 40 cm apart.
We tried to adjust the sensitivity of the receiver so that the meter on top read 1.0, but it only went up
to 0.9. Then, we moved the units apart and recorded the intensity readings at various distances.

\begin{table}[h!]
\centering
\caption{Intensity vs. Distance}
\label{Ixr}
\begin{tabular}{|r|l|}
\hline
Distance ($d$) & Intensity reading ($I$) \\ \hline
40 cm          & 0.9                     \\ \hline
50 cm          & 0.55                    \\ \hline
60 cm          & 0.36                    \\ \hline
70 cm          & 0.28                    \\ \hline
80 cm          & 0.14                    \\ \hline
90 cm          & 0.08                    \\ \hline
100 cm         & 0.06                    \\ \hline
\end{tabular}
\end{table}

\pagebreak

Finally, we set $d$ to a value between 70 cm and 90 cm and slowly decreased $d$. We observed that the meter
deflection increased as the distance decreased.

\subsection{Reflection}

We placed a metal plate (the ``reflector'') into the space between the transmitter and receiver below their ``eye level.''
The intensity reading on the receiver increased.

\subsection{Polarization}

We rotated the roll of the receiver relative to the transmitter and observed what happened to the intensity reading.
As we increased the roll from zero to 90 degrees, the intensity decreased, and at 90 degrees, it was zero.

\subsection{Signal distribution}

We changed the angle of the goniometer, rotating the transmitter and receiver relative to the center of the
goniometer, and took intensity readings at each angle setting.

\begin{table}[h!]
\centering
\caption{Intensity vs. angle}
\label{Ixt}
\begin{tabular}{|r|l|}
\hline
Angle        & Intensity reading \\ \hline
$0^{\circ}$  & 0.9               \\ \hline
$10^{\circ}$ & 0.76              \\ \hline
$20^{\circ}$ & 0.06              \\ \hline
$30^{\circ}$ & 0                 \\ \hline
$40^{\circ}$ & 0                 \\ \hline
$50^{\circ}$ & 0                 \\ \hline
$60^{\circ}$ & 0                 \\ \hline
$70^{\circ}$ & 0                 \\ \hline
$80^{\circ}$ & 0                 \\ \hline
$90^{\circ}$ & 0                 \\ \hline
\end{tabular}
\end{table}

\pagebreak

\subsection{Law of reflection}

We stood up the reflector on the center of the goniometer and set the angle of incidence to $theta$ by
rotating the reflector till its surface normal was $theta$ relative to the transmitter arm of the
goniometer. Then, we found the angle between the receiver and the surface normal of the reflector by
rotating the receiver arm of the goniometer until the maximum intensity reading was reached.

\begin{table}[h!]
\centering
\caption{Angle of incidence vs. angle of reflection}
\label{angle-of-reflection}
\begin{tabular}{|l|l|}
\hline
Angle of incidence & Angle of reflection \\ \hline
$10^{\circ}$       & $40^{\circ}$        \\ \hline
$20^{\circ}$       & $27^{\circ}$        \\ \hline
$30^{\circ}$       & $36^{\circ}$        \\ \hline
$40^{\circ}$       & $41^{\circ}$        \\ \hline
$50^{\circ}$       & $50^{\circ}$        \\ \hline
$60^{\circ}$       & $58^{\circ}$        \\ \hline
$70^{\circ}$       & $87^{\circ}$        \\ \hline
$80^{\circ}$       & $84^{\circ}$        \\ \hline
$90^{\circ}$       & $90^{\circ}$        \\ \hline
\end{tabular}
\end{table}



\section{Discussion}

How well do the data in Table 1 adhere to the inverse square law? We found out by multiplying the
intensity reading by the square of the distance and finding the mean and standard deviation of
the results:

\begin{table}[h!]
\centering
\begin{tabular}{|r|l|r|}
\hline
Distance ($d$) & Intensity reading ($I$) & $Id^2$      \\ \hline
40 cm          & 0.9                     & 1440 cm$^2$ \\ \hline
50 cm          & 0.55                    & 1375 cm$^2$ \\ \hline
60 cm          & 0.36                    & 1296 cm$^2$ \\ \hline
70 cm          & 0.28                    & 1372 cm$^2$ \\ \hline
80 cm          & 0.14                    &  896 cm$^2$ \\ \hline
90 cm          & 0.08                    &  648 cm$^2$ \\ \hline
100 cm         & 0.06                    &  600 cm$^2$ \\ \hline
\end{tabular}
\end{table}

The mean value of $Id^2$ is 1090 cm$^2$, and the standard deviation is 365 cm$^2$, which is
33\% of the mean value. The value of $Id^2$ decreases as $d$ increases, so maybe the intensity
measurements become less accurate as they decrease. These results confirm the accuracy of
the inverse square law.

In the reflection experiment (Section 2.2), we found that the receiver picks up more radiation when a
reflective surface is placed between the transmitter and the receiver. This is probably because the
radiation reflects back up towards the receiver instead of escaping into space.

In Section 2.3, we found that the intensity of incoming radiation decreases as the roll $\theta$ of the
receiver relative to the transmitter increases to $90^\circ$. This may be because the waves coming out
of the transmitter are polarized in one plane and then are polarized in another plane angled $\theta$
to the plane of the transmitter as they enter the receiver. The intensity of the received radiation is
$\cos^2 \theta$ times the intensity of the emitted radiation.

In Section 2.4, most of the energy of the wave is concentrated in front of the transmitter, and the
portion of the wave outside of the central $60^\circ$ arc is rarefied.

In Section 2.5, the angle of reflection correlates positively with the angle of incidence except when
the angle of incidence is $10^\circ$ and $70^\circ$. From $10^\circ$ to $20^\circ$ and from $70^\circ$
to $80^\circ$, the observed angle of reflection \textit{decreases}. This may be caused by the geometry
of the transmitter and receiver: each has a conical waveguide that directs incoming or outgoing light.
At these angles, the waveguides may increase the amount of radiation detected, as in Section 2.2.

\section{Conclusion}

Electromagnetic radiation obeys an inverse-square law for intensity, and polarization decreases its intensity.
Reflection increases the local intensity of the light around the reflective surface.

\end{document}
