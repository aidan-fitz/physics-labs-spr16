%%%%%%%%%%Packages%%%%%%%%%

\documentclass[11pt, titlepage, letterpaper, twoside]{article}
\usepackage{amsmath,amsthm,amssymb}
\usepackage{hyperref, pgf, tikz}
\usepackage{fancyhdr}
\usetikzlibrary{arrows}
\usepackage[margin=1.25in]{geometry}

\usetikzlibrary{circuits,circuits.ee.IEC}
\usepackage{circuitikz}
\tikzset{circuit declare symbol = amm}
\tikzset{set amm graphic ={draw,generic circle IEC, minimum size=7mm,info=center:A}}
\tikzset{circuit declare symbol = voltm}
\tikzset{set voltm graphic ={draw,generic circle IEC, minimum size=7mm,info=center:V}}

%%%%%%%%%%%%%%%%%%%%%%%%%%%
\pagestyle{fancy}

\frenchspacing

\lhead{Lab \#2 }  %insert lab # here
\rhead{\thepage}
\cfoot{}

\title{\textbf{Joule Heating} \\ \ \\ \large Lab \#2 }
\author{Name: Aidan Fitzgerald \\ Partners: Margaret Burkart and Kristi Fok}
\date{May 31, 2016}
\begin{document}

\maketitle

\begin{center}
\LARGE Joule Heating
\end{center}

\section*{Objective}
Measure Joule heating experimentally.

\section{Introduction}

Joule heating is a process by which electrical energy is converted to thermal energy. It occurs because conductors (excluding superconductors) have resistance,
which means that a constant supply of power must be added just to maintain a constant current. The relationship between input power, voltage, and current
is given by

\begin{equation}
  P = IV = I^2 R = \frac{V^2}{R}
\end{equation}

Multiplying through by time, $t$, we obtain an equation for the work done on all the charges in the conductor over a period of time to produce a current:

\begin{equation}
  W = IVt
\end{equation}

Ideally, the work done on the electric charges should equal the heat generated by the current, $W = \Delta H$, because of conservation of energy. The Joule heat
produced by the wire can be measured using a calorimeter, according to the diagram below:

\begin{figure}[h!]
\begin{center}
\begin{circuitikz}[circuit ee IEC] \draw
(0,0) to[battery,l=12 V] (0,4)
to[vR,l=Rheostat] (4,4)
to[amm] (8,4) -- (8,0) -- (5,0) -- (5,-2) to[R,l=Immersion Heater] (3,-2) -- (3,0) -- (0,0)
(2.5,0) -- (2.5,1) to[voltm] (5.5,1) -- (5.5,0)
;
\end{circuitikz}
\end{center}
\caption{Calorimeter setup}
\end{figure}

When the heat is measured using a calorimeter, it is given by\nopagebreak
\begin{equation}
  \Delta H = (m_wc_w + m_{cal}c_{cal} + m_{coil}c_{coil})(T_f - T_i)
\end{equation}
\nopagebreak \noindent where $(m_w, c_w)$, $(m_{cal}, c_{cal})$, and $(m_{coil}, c_{coil})$ are the mass and specific heat of the water, calorimeter cup, and coil, and $T_i$ and $T_f$
are the initial and final temperatures of the system.


\section{Procedures and Results}

We built the setup shown in Figure 1 and recorded the masses and specific heats of the calorimeter cup, coil, and water.

\newcommand{\degC}{$^\circ\mathrm{C}$}
\newcommand{\JgK}{$\mathrm{J / (g \cdot {^\circ\mathrm{C}}) }$}
\newcommand{\water}{$\mathrm{H_2O}$}

\begin{center}
\begin{tabular}{| r | r | r | r |}
  \hline
                  & Mass    & Material & Specific heat \\ \hline
  Calorimeter cup & 105.3 g & Copper   & 0.385 \JgK    \\ \hline
  Immersion coil  & 1.4 g   & Nichrome & 0.107 \JgK    \\ \hline
  Calorimeter cup
  and water       & 254.4 g &          &               \\ \hline
  Water           & 149.1 g & \water   & 4.186 \JgK    \\ \hline
\end{tabular}
\end{center}

We filled the calorimeter cup two-thirds full with cool tap water and measured the temperature of the system to be 25.2 \degC. We set the input voltage to 12 V
and turned on the power supply, adjusting the rheostat until the current was between 2 and 3 A. We took current and voltage readings every 60 s. After 300 s,
the temperature of the system had risen to 30.1 \degC.

\begin{center}
\begin{tabular}{| r | r | r |}
  \hline
  Time  & Voltage & Current \\ \hline
  0 s   & 5.46 V  & 2.66 A  \\ \hline
  60 s  & 5.32 V  & 2.69 A  \\ \hline
  120 s & 5.30 V  & 2.60 A  \\ \hline
  180 s & 4.16 V  & 2.51 A  \\ \hline
  240 s & 4.41 V  & 2.39 A  \\ \hline
  300 s & 4.41 V  & 2.22 A  \\ \hline
\end{tabular}
\end{center}

\section{Discussion}

The averages of the voltage and current measurements are 4.84 V and 2.51 A. Substituting into Equation 2, we obtain the work done to produce the electric
current:

\begin{align*}
W &= IVt \\
  &= (2.51\,\mathrm{A})(4.84\,\mathrm{V})(300\,\mathrm{s}) \\
  &= 3640\,\mathrm{J}.
\end{align*}

Substituting the mass measurements, accepted specific heat values, and initial and final temperatures into Equation 3, the heat generated by the electric
current in the calorimeter is:

\newcommand{\degCm}{{^\circ\mathrm{C}}}
\newcommand{\JgCm}{\mathrm{J / (g \cdot \degCm) }}

\begin{align*}
\Delta H &= (m_wc_w + m_{cal}c_{cal} + m_{coil}c_{coil})(T_f - T_i) \\
         &= (149.1\,\mathrm{g} \cdot 4.186\,\JgCm + 105.3\,\mathrm{g} \cdot 0.385\,\JgCm + 1.4\,\mathrm{g} \cdot 0.107\,\JgCm)(30.1 \degCm - 25.2 \degCm) \\
         &= 3260\,\mathrm{J}.
\end{align*}

The percent difference between $W$ and $\Delta H$ is
$$
\mathrm{\%\,diff} = \frac{\left| W - \Delta H \right|}{\frac{W + \Delta H}{2}} = \frac{380\,\mathrm{J}}{3450\,\mathrm{J}} = 11\%.
$$

The rheostat cannot account for this inconsistency because the voltmeter only runs parallel to the immersion heater. However, the presence of the voltmeter
itself may account for it: because the voltmeter is not an ideal voltmeter, some current is allowed to flow through it, so some heat is dissipated through
the voltmeter instead of the heating coil. That heat loss is given by $W - \Delta H = 380\,\mathrm{J}$.

\section{Conclusion}

The heat generated by a live conductor with nonzero resistance, $\Delta H$, is equal to the work done on the charges in the conductor, $W$.


\end{document}
